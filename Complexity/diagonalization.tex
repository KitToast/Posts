\documentclass[12pt]{article}
\setlength{\oddsidemargin}{0in}
\setlength{\evensidemargin}{0in}
\setlength{\textwidth}{6.5in}
\setlength{\parindent}{0in}
\setlength{\parskip}{\baselineskip}

\usepackage{amsmath,amsfonts,amssymb,amsthm}

\newtheorem{theorem}{Theorem}[section]
\newtheorem{question}{Question}[section]
\newtheorem{lemma}[theorem]{Lemma}
\newtheorem{definition}{Definition}[section]
\newtheorem{proposition}{Proposition}[section]

\title{Diagonalization}
\author{Edward Kim}

\begin{document}
\maketitle
\hrulefill

Diagonalization is a time-tested tool dating back to Cantor's famous proof on the cardinality of $\mathbb{R}$. The of the undecidability of halting problem uses diagonalization to show that no TM can decide whether an arbitrary TM will halt on a given input $x$. This technique will be used to show some basic results on the natural hierarchy of deterministic and non-deterministic TMs based on the amount of resources one allocates to it.

\section{The Time Hierarchy Theorems} 

\begin{definition}
Given a function $f: \mathbb{N} \rightarrow \mathbb{N}$, we say that $f$ is time-constructible if the map $x \mapsto f(|x|)$ can be computed in $O(f(|x|))$ time
\end{definition}

\begin{theorem}
Let $f,g$ be time-constructible functions such that $f(n)\log f(n) = o(g(n))$, then
$$ \boldmath{DTIME}(f(n)) \subset \boldmath{DTIME}(g(n))$$
\end{theorem}	

\begin{proof}
	We assume that simulating a machine by universal TM, $\mathcal{U}$ is at most logarithmic. Consider the language $L$ determined by the machine $D$ which exhibits the following behavior: on input $x$, run the TM $M_x(x)$ for $g(|x|)$ steps. If the machine halts at any point with output $y$, output its negation $\neg y$. Otherwise, output 0. \par By construction, $L \in \boldmath{DTIME}(g(n))$. We shall prove that $L$ cannot be decided by any machine in $\boldmath{DTIME}(f(n))$. For the sake of contradiction, suppose that there exists machine $G$ such that $G$ decides $L$ and for any input $x$, there exists a constant $c$ such that $G$ runs for at most $c|x|$. By our assumption above, we can simulate $G$ on $\mathcal{U}$ with $c|x|\log|x|$ steps. Since $f(n)\log f(n) = o(g(n))$, there exists an string $y$ such that $g(|y|) > f(|y|) \log |y|$ such that $M_y = G$ (this follows from the padding lemma). Hence, $D(y) = \neg b = G(y)$. This gives us our contradiction.
\end{proof}

\begin{theorem}
If $f,g$ are time-constructible functions such that $f(n+1) = o(g(n))$, then
$$ \boldmath{NTIME}(f(n)) \subset \boldmath{NTIME}(g(n)) $$
\end{theorem}

\end{document}